\chapter{Differential Theory}
\section{Topology}
\begin{definition}
A topological space is a set $S$ together with a collect $\mathcal{O}$ of subsets called open sets s.t.
\begin{enumerate}
\item[(T1)] $\emptyset\in \mathcal{O}$ and $S\in \mathcal{O}$;
\item[(T2)] If $U_1,U_2\in \mathcal{S}$, then $U_1\cap U_2 \in \mathcal{O}$;
\item[(T3)] The union of any collection of open sets is open.
\end{enumerate}
\end{definition}
For such a topological space the \textbf{closed sets} are the elements of $\gamma=\{A\mid A^c\in \mathcal{O}\}$.
An \textbf{open neighborhood of a point} $u$ is a topological space $S$ is an open set $U$ s.t. $u\in U$. Similarly,
for a subset $A$ of $S$, $U$ is an \textbf{open neighborhood} of $A$ if $U$ is open and $A\subset U$. If $A$ is a
subset of a topological space $S$, the \textbf{relative topology} on $A$ is defined by $\mathcal{O}_A=\{U\cap A\mid
U\in\mathcal{O}\}$. 

Then a \tb{basis} for the topology is a collection $\mca{B}$ of opensets s.t. every open set of $S$ is a union of 
elements of $\mca{B}$. The topology is called \tb{first coutable} if for each $u\in S$, there is a countable collection
$\{U_n\}$ of neighborhoods of $u$ s.t. for any neighborhood $U$ of $u$, there is an $n$ so $U_n \subset U$. The
topology is called \tb{second countable} if it has a countable basis. 

Let $\{u_n\}$ be a sequence of points in $S$. The sequence is said to \tb{coverge} if there is a point $u\in S$ s.t.
for every neighborhood $U$ of $u$, there is an $N$ s.t. $n\geq N$ implies $u_n\in U$. We say that $\{u_n\}$ converges
to $u$ or $u$ is a \tb{limit point} of $\{u_n\}$.
\begin{example}
The standard topology of $\mathbb{R}$ is the unions of open intervals $(a,b)$. Then  $\mathbb{R}$ is second countable
(hence first countable) with a basis $$\left\lbrace\left(r_n-\frac{1}{m},r_n+\frac{1}{m}\right)\mid r_n\in\mathbb{Q},m\in 
\mathbb{N}^+\right\rbrace$$.
\end{example}
\begin{definition}
Let $S$ be a topological space and $A\subset S$. Then the \tb{closure} of $A$, denoted $\overline{A}$ is the inersection
of all colsed sets containing $A$. The \tb{interior} of $A$, denoted $\mathring{A}$ is the union of all open sets contained 
in $A$. The \tb{boundary} of $A$, denoted $\partial A := \overline{A}\cap \overline{A^c}$.
\end{definition}
Thus, $\partial A$ is closed, and $\partial A = \partial A^c$. Note that $A$ is open iff $A=\mathring{A}$ and closed iff
$A=\overline{A}$.
\begin{prop}
Let S be a topological space  and $A\subset S$.
\begin{enumerate}
\item[(i)] $u\in \overline{A}$ iff for every neighborhood U of u, $U\cap A\neq \emptyset$.
\item[(ii)] $u\in\mathring{A}$ iff there is a neighborhood U of u s.t. $U\subset A$.
\item[(iii)] $u\in\partial A$ iff for every neighborhood U of u, $U\cap A\neq\emptyset$ and $U\cap A^c\neq\emptyset$.
\end{enumerate}
\end{prop}
\begin{definition}
A point $u\in S$ is called \tb{isolated} iff $\{u\}$ is open. A subset A of S is called \tb{dense} in S iff $\overline{A}=S$
and is called \tb{nowhere dense} iff $(\overline{A})^c$ is dense in S. Thus, A is nowhere dense iff $\mathring{\overline{A}}
=\emptyset$.
\end{definition}
\begin{definition}
A topological space S is called \tb{Hausdorff} iff each two distinct points have disjoint neighborhoods. Similarly, S is called
\tb{normal} iff each two disjoint closed sets have disjoint neighborhoods.
\end{definition}
\begin{prop}
\begin{enumerate}
\item[(i)] A space S is Hausdorff iff $\Delta_S=\lbrace(u,u)\mid u\in S\rbrace$ is closed in $S\times S$ is the product topology.
\item[(ii)] A first countable space S is Hausdorff iff all sequences have at most one limit point. 
\end{enumerate}
\end{prop}
\begin{proof}
If $\Delta_S$ is closed and $u_1,u_2$ are distinct, there is an open rectangle $U\times V$ containing $(u_1,u_2)$ and $U\times V
\subset \Delta_S^c$. Then in S, U and V are disjoint because if $\exists p\in U\cap V$, then $(p,p)\in U\times V$ it countered 
with closed set $\Delta_S$. The converse is similar and we leave it as an exercise.
\end{proof}
\begin{definition}
Let $\R{+}$ denote the nonnegative real numbers with a point $\{+\infty\}$ adjoined, and topology generated by the open intervals
of the form $(a,b)$. A \tb{metric} on set M is a function $d:M\times M\rightarrow \R{+}$ s.t.
\begin{enumerate}
\item[(M1)] $d(m_1,m_2)=0$ iff $m_1=m_2$;
\item[(M2)] $d(m_1,m_2)=d(m_2,m_1)$;
\item[(M3)] $d(m_1,m_3)\leq d(m_1,m_2)+d(m_2,m_3)$
\end{enumerate}
\end{definition}
For $\varepsilon > 0$, the $\varepsilon$ \tb{disk} about m is defined by $D_\varepsilon(m)=\lbrace m'\in M\mid d(m',m)<\varepsilon$.
The collection of subsets of M that are unions of such disks is the metric topology of the metric space (M,d). Two metrics on a set
are called \tb{equivalent} if they induce the same metric topology. $\{u_n\}$ is a \tb{Cauchy sequence} iff for all $\varepsilon>0$,
there is an $N\in \N$ s.t. $n,m\geq N$ implies $d(u_n,u_m)<\varepsilon$. The space M is called \tb{complete} if every Cauchy sequence 
converges. We define $d(u,A)=inf\lbrace d(u,v)\mid v\in A \rbrace$ and $d(u,\emptyset)=\infty$.
\begin{prop}
Every metric space is normal.
\end{prop}
\begin{proof}
Let A and B be closed, disjoint subsets of M, and let 
\begin{align*}
U &= \lbrace u\in M\mid d(u,A)<d(u,B)\rbrace \\
V &= \lbrace v\in M\mid d(v,A)>d(v,B)\rbrace
\end{align*}
It is verified that U and V are open, disjoint and $A\subset U, B \subset V$.
\end{proof}
\begin{definition}
If $\varphi:S\rightarrow T$ is \tb{continuous} at $u\in S$ if $\forall V\ni\varphi(u),\exists U\ni u \Rightarrow \varphi(U)\subset V$.
If $\forall V\subset T, \varphi^{-1}(V)=\lbrace u\in S\mid \varphi(u)\in V$ is open in S, $\varphi$ is \tb{continuous}.
\end{definition}
If $\varphi:S\rightarrow T$ is a \tb{bijection}, $\varphi$ and $\varphi^{-1}$ are continuous, then $\varphi$ is a \tb{homeomorphism}
and S and T are \tb{homeomorphic}.
\begin{prop}
$\varphi$ is continuous iff $\forall A\subset S \Rightarrow \varphi(\overline{A})\subset \overline{\varphi(A)}$.
\end{prop}
\begin{proof}
If $\varphi$ is continuous, then $\varphi^{-1}(\overline{\varphi(A)})$ is closed. But $A\subset \varphi^{-1}(\overline{\varphi(A)})$
and hence $\overline{A}\subset\varphi^{-1}(\overline{\varphi(A)})$ or $\varphi(\overline{A})\subset \overline{\varphi(A)}$. Conversely,
let $B\subset T$ be closed and $A=\varphi^{-1}(B)$. Then $\overline{(A)}\subset \varphi^{-1}(B)=A$, so A is closed.
\end{proof}
\begin{prop}
Let $\varphi:S\rightarrow T$ and S first countable set. Then $\varphi$ is continuous iff $\forall u_n\rightarrow u \Rightarrow \varphi(u_n)
\rightarrow \varphi(u)$.
\end{prop}
\begin{prop}
Let M and N be metric spaces with N complete. Then the collection $C(M,N)$ of all continuous maps $\varphi:M\rightarrow N$ forms a 
complete metric space with the metric $d^0(\varphi,\phi)=sup\lbrace d(\varphi(u),\phi(u))\mid u\in M\rbrace$.
\end{prop}
\begin{proof}
It is readily verified that $d^0$ is a metric. Convergence of sequence $f_n\in C(M,N)$ to $f\in C(M,N)$ in the metric $d^0$ is the
same as uniform convergence, that is, for all $\varepsilon > 0$ there is an N s.t. if $n\geq N$, $d(f_n(x),f(x))\leq \varepsilon$ for
all $x\in M$. If $f_n$ is a Cauchy sequence in $C(M,N)$, then since $d(f_n(x),f_m(x))\leq d^0(f_n,f_m)$. $f_n(x)$ is Cauchy for each
point $x\in M$. Thus $f_n$ converges pointwise, define a function $f(x)$. We must show that $f_n\rightarrow f$ uniformly and that $f$
is continuous. First of all, given $\varepsilon > 0$, choose N s.t. $d^0(f_n,f_m)<\varepsilon/2$ if $n,m\geq N$. Then for any $x\in M$,
pick $N_x\geq N$ s.t. $d(f_m(x),f(x))<\varepsilon/2$ if $m\geq N_x$. Thus with $n\geq N$ and $m\geq N_x$, $ d(f_n(x),f(x))\leq d(f_n(x),
f_m(x)) + d(f_m(x),f(x))<\varepsilon/2+\varepsilon/2=\varepsilon$ So $f_n\rightarrow f$ uniformly.
\end{proof}
\begin{definition}
S is called \tb{compact} iff $\forall \cup_\alpha U_\alpha = S$ there is a finite subcovering. A subset $A\subset S$ is called \tb{compact}
iff A is compact in the relative topology. A space is called $locally compact$ iff each point has a neighborhood whose closure is compact.
\end{definition}
\begin{thm}[\tb{Boizano-Weierstrass}]
If S is a first countable space and is compact, then every sequence has a convergent subsequence.
\end{thm}
\begin{proof}
Suppose $\{u_n\}$ contains no convergent subsequences. Then we may assume all points are distinct. Each $u_n$ has a neighborhood $\mca{O}_n$
that contains no other $u_m$. $\{u_n\}$ is closed, so that $\mca{O}_n$ together with $\{u_n\}^c$ forms an open covering of S, with no finit subcovering.
\end{proof}
In a metric space, every compact subset is closed and bounded(Heine Borel theorem).
\begin{prop}
Let S be a Hausdorff space. Then every compact subset of S is closed. Also, every compact Hausdorff space is normal.
\end{prop}
\begin{proof}
Let $u\in A^c$ and $v\in A$, where A is compact in S. There are disjoint neighborhoods of u and v and, since A is compact, there are disjoint 
neighborhood of u and A. Thus $A^c$ is open. The second part is an exercise.
\end{proof}
\begin{prop}
Let S be a Hausdorff space that is locally homeomorphic to a locally compact Hausdorff space. Then S is locally compact.
\end{prop}
\begin{proof}
Let $U\subset S$ be homeomorphic to $\varphi(U)\subset T$. There is a neighborhood $V$ of $\varphi(u)$ so $\overline{V}\subset \varphi(U)$ and
$\overline{V}$ is compact. Then $\varphi^{-1}(\overline{V})$ is compact, and hence closed in S. $\varphi^{-1}(\overline{V})\subset 
\overline{\varphi^{-1}(V)}$. Thus $\varphi^{-1}(V)$ has compact closure.
\end{proof}
\section*{Exercises}
\section{Finite-Dimensional Banach Sapce}
\section*{Exercises}
\section{Local Differential Calculus}
\section*{Exercises}
\section{Manifolds and Mappings}
\section*{Exercises}
\section{Vector Bundles}
\section*{Exercises}
\section{The Tangent Bundle}
\section*{Exercises}
\section{Tensors}
\section*{Exercises}