\chapter{Differential Theory}
\section{Topology}
\begin{definition}
A topological space is a set $S$ together with a collect $\mathcal{O}$ of subsets called open sets s.t.
\begin{enumerate}
\item[(T1)] $\emptyset\in \mathcal{O}$ and $S\in \mathcal{O}$;
\item[(T2)] If $U_1,U_2\in \mathcal{S}$, then $U_1\cap U_2 \in \mathcal{O}$;
\item[(T3)] The union of any collection of open sets is open.
\end{enumerate}
\end{definition}
For such a topological space the \textbf{closed sets} are the elements of $\gamma=\{A\mid A^c\in \mathcal{O}\}$.
An \textbf{open neighborhood of a point} $u$ is a topological space $S$ is an open set $U$ s.t. $u\in U$. Similarly,
for a subset $A$ of $S$, $U$ is an \textbf{open neighborhood} of $A$ if $U$ is open and $A\subset U$. If $A$ is a
subset of a topological space $S$, the \textbf{relative topology} on $A$ is defined by $\mathcal{O}_A=\{U\cap A\mid
U\in\mathcal{O}\}$. 

Then a \tb{basis} for the topology is a collection $\mca{B}$ of opensets s.t. every open set of $S$ is a union of 
elements of $\mca{B}$. The topology is called \tb{first coutable} if for each $u\in S$, there is a countable collection
$\{U_n\}$ of neighborhoods of $u$ s.t. for any neighborhood $U$ of $u$, there is an $n$ so $U_n \subset U$. The
topology is called \tb{second countable} if it has a countable basis. 

Let $\{u_n\}$ be a sequence of points in $S$. The sequence is said to \tb{coverge} if there is a point $u\in S$ s.t.
for every neighborhood $U$ of $u$, there is an $N$ s.t. $n\geq N$ implies $u_n\in U$. We say that $\{u_n\}$ converges
to $u$ or $u$ is a \tb{limit point} of $\{u_n\}$.
\begin{example}
The standard topology of $\mathbb{R}$ is the unions of open intervals $(a,b)$. Then  $\mathbb{R}$ is second countable
(hence first countable) with a basis $$\left\lbrace\left(r_n-\frac{1}{m},r_n+\frac{1}{m}\right)\mid r_n\in\mathbb{Q},m\in 
\mathbb{N}^+\right\rbrace$$.
\end{example}
\begin{definition}
Let $S$ be a topological space and $A\subset S$. Then the \tb{closure} of $A$, denoted $\overline{A}$ is the inersection
of all colsed sets containing $A$. The \tb{interior} of $A$, denoted $\mathring{A}$ is the union of all open sets contained 
in $A$. The \tb{boundary} of $A$, denoted $\partial A := \overline{A}\cap \overline{A^c}$.
\end{definition}
Thus, $\partial A$ is closed, and $\partial A = \partial A^c$. Note that $A$ is open iff $A=\mathring{A}$ and closed iff
$A=\overline{A}$.
\begin{prop}
Let S be a topological space  and $A\subset S$.
\begin{enumerate}
\item[(i)] $u\in \overline{A}$ iff for every neighborhood U of u, $U\cap A\neq \emptyset$.
\item[(ii)] $u\in\mathring{A}$ iff there is a neighborhood U of u s.t. $U\subset A$.
\item[(iii)] $u\in\partial A$ iff for every neighborhood U of u, $U\cap A\neq\emptyset$ and $U\cap A^c\neq\emptyset$.
\end{enumerate}
\end{prop}
\begin{definition}
A point $u\in S$ is called \tb{isolated} iff $\{u\}$ is open. A subset A of S is called \tb{dense} in S iff $\overline{A}=S$
and is called \tb{nowhere dense} iff $(\overline{A})^c$ is dense in S. Thus, A is nowhere dense iff $\mathring{\overline{A}}
=\emptyset$.
\end{definition}
\begin{definition}
A topological space S is called \tb{Hausdorff} iff each two distinct points have disjoint neighborhoods. Similarly, S is called
\tb{normal} iff each two disjoint closed sets have disjoint neighborhoods.
\end{definition}
\begin{prop}
\begin{enumerate}
\item[(i)] A space S is Hausdorff iff $\Delta_S=\lbrace(u,u)\mid u\in S\rbrace$ is closed in $S\times S$ is the product topology.
\item[(ii)] A first countable space S is Hausdorff iff all sequences have at most one limit point. 
\end{enumerate}
\end{prop}
\begin{proof}
If $\Delta_S$ is closed and $u_1,u_2$ are distinct, there is an open rectangle $U\times V$ containing $(u_1,u_2)$ and $U\times V
\subset \Delta_S^c$. Then in S, U and V are disjoint because if $\exists p\in U\cap V$, then $(p,p)\in U\times V$ it countered 
with closed set $\Delta_S$. The converse is similar and we leave it as an exercise.
\end{proof}
\begin{definition}
Let $\R{+}$ denote the nonnegative real numbers with a point $\{+\infty\}$ adjoined, and topology generated by the open intervals
of the form $(a,b)$. A \tb{metric} on set M is a function $d:M\times M\rightarrow \R{+}$ s.t.
\begin{enumerate}
\item[(M1)] $d(m_1,m_2)=0$ iff $m_1=m_2$;
\item[(M2)] $d(m_1,m_2)=d(m_2,m_1)$;
\item[(M3)] $d(m_1,m_3)\leq d(m_1,m_2)+d(m_2,m_3)$
\end{enumerate}
\end{definition}
For $\varepsilon > 0$, the $\varepsilon$ \tb{disk} about m is defined by $D_\varepsilon(m)=\lbrace m'\in M\mid d(m',m)<\varepsilon$.
The collection of subsets of M that are unions of such disks is the metric topology of the metric space (M,d). Two metrics on a set
are called \tb{equivalent} if they induce the same metric topology. $\{u_n\}$ is a \tb{Cauchy sequence} iff for all $\varepsilon>0$,
there is an $N\in \N$ s.t. $n,m\geq N$ implies $d(u_n,u_m)<\varepsilon$. The space M is called \tb{complete} if every Cauchy sequence 
converges. We define $d(u,A)=inf\lbrace d(u,v)\mid v\in A \rbrace$ and $d(u,\emptyset)=\infty$.
\begin{prop}
Every metric space is normal.
\end{prop}
\begin{proof}
Let A and B be closed, disjoint subsets of M, and let 
\begin{align*}
U &= \lbrace u\in M\mid d(u,A)<d(u,B)\rbrace \\
V &= \lbrace v\in M\mid d(v,A)>d(v,B)\rbrace
\end{align*}
It is verified that U and V are open, disjoint and $A\subset U, B \subset V$.
\end{proof}
\begin{definition}
If $\varphi:S\rightarrow T$ is \tb{continuous} at $u\in S$ if $\forall V\ni\varphi(u),\exists U\ni u \Rightarrow \varphi(U)\subset V$.
If $\forall V\subset T, \varphi^{-1}(V)=\lbrace u\in S\mid \varphi(u)\in V$ is open in S, $\varphi$ is \tb{continuous}.
\end{definition}
If $\varphi:S\rightarrow T$ is a \tb{bijection}, $\varphi$ and $\varphi^{-1}$ are continuous, then $\varphi$ is a \tb{homeomorphism}
and S and T are \tb{homeomorphic}.
\begin{prop}
$\varphi$ is continuous iff $\forall A\subset S \Rightarrow \varphi(\overline{A})\subset \overline{\varphi(A)}$.
\end{prop}
\begin{proof}
If $\varphi$ is continuous, then $\varphi^{-1}(\overline{\varphi(A)})$ is closed. But $A\subset \varphi^{-1}(\overline{\varphi(A)})$
and hence $\overline{A}\subset\varphi^{-1}(\overline{\varphi(A)})$ or $\varphi(\overline{A})\subset \overline{\varphi(A)}$. Conversely,
let $B\subset T$ be closed and $A=\varphi^{-1}(B)$. Then $\overline{(A)}\subset \varphi^{-1}(B)=A$, so A is closed.
\end{proof}
\begin{prop}
Let $\varphi:S\rightarrow T$ and S first countable set. Then $\varphi$ is continuous iff $\forall u_n\rightarrow u \Rightarrow \varphi(u_n)
\rightarrow \varphi(u)$.
\end{prop}
\begin{prop}
Let M and N be metric spaces with N complete. Then the collection $C(M,N)$ of all continuous maps $\varphi:M\rightarrow N$ forms a 
complete metric space with the metric $d^0(\varphi,\phi)=sup\lbrace d(\varphi(u),\phi(u))\mid u\in M\rbrace$.
\end{prop}
\begin{proof}
It is readily verified that $d^0$ is a metric. Convergence of sequence $f_n\in C(M,N)$ to $f\in C(M,N)$ in the metric $d^0$ is the
same as uniform convergence, that is, for all $\varepsilon > 0$ there is an N s.t. if $n\geq N$, $d(f_n(x),f(x))\leq \varepsilon$ for
all $x\in M$. If $f_n$ is a Cauchy sequence in $C(M,N)$, then since $d(f_n(x),f_m(x))\leq d^0(f_n,f_m)$. $f_n(x)$ is Cauchy for each
point $x\in M$. Thus $f_n$ converges pointwise, define a function $f(x)$. We must show that $f_n\rightarrow f$ uniformly and that $f$
is continuous. First of all, given $\varepsilon > 0$, choose N s.t. $d^0(f_n,f_m)<\varepsilon/2$ if $n,m\geq N$. Then for any $x\in M$,
pick $N_x\geq N$ s.t. $d(f_m(x),f(x))<\varepsilon/2$ if $m\geq N_x$. Thus with $n\geq N$ and $m\geq N_x$, $ d(f_n(x),f(x))\leq d(f_n(x),
f_m(x)) + d(f_m(x),f(x))<\varepsilon/2+\varepsilon/2=\varepsilon$ So $f_n\rightarrow f$ uniformly.
\end{proof}
\begin{definition}
S is called \tb{compact} iff $\forall \cup_\alpha U_\alpha = S$ there is a finite subcovering. A subset $A\subset S$ is called \tb{compact}
iff A is compact in the relative topology. A space is called $locally compact$ iff each point has a neighborhood whose closure is compact.
\end{definition}
\begin{thm}[\tb{Boizano-Weierstrass}]
If S is a first countable space and is compact, then every sequence has a convergent subsequence.
\end{thm}
\begin{proof}
Suppose $\{u_n\}$ contains no convergent subsequences. Then we may assume all points are distinct. Each $u_n$ has a neighborhood $\mca{O}_n$
that contains no other $u_m$. $\{u_n\}$ is closed, so that $\mca{O}_n$ together with $\{u_n\}^c$ forms an open covering of S, with no finit subcovering.
\end{proof}
In a metric space, every compact subset is closed and bounded(Heine Borel theorem).
\begin{prop}
Let S be a Hausdorff space. Then every compact subset of S is closed. Also, every compact Hausdorff space is normal.
\end{prop}
\begin{proof}
Let $u\in A^c$ and $v\in A$, where A is compact in S. There are disjoint neighborhoods of u and v and, since A is compact, there are disjoint 
neighborhood of u and A. Thus $A^c$ is open. The second part is an exercise.
\end{proof}
\begin{prop}
Let S be a Hausdorff space that is locally homeomorphic to a locally compact Hausdorff space. Then S is locally compact.
\end{prop}
\begin{proof}
Let $U\subset S$ be homeomorphic to $\varphi(U)\subset T$. There is a neighborhood $V$ of $\varphi(u)$ so $\overline{V}\subset \varphi(U)$ and
$\overline{V}$ is compact. Then $\varphi^{-1}(\overline{V})$ is compact, and hence closed in S. $\varphi^{-1}(\overline{V})\subset 
\overline{\varphi^{-1}(V)}$. Thus $\varphi^{-1}(V)$ has compact closure.
\end{proof}
\begin{definition}
A covering $\{U_\alpha\}$ of S is called a \tb{refinement} of a covering $\{V_i\}$ iff $\forall U_\alpha$ there is a $V_i$ s.t. $U_\alpha\subset
V_i$. A covering $\{U_\alpha\}$ of S is called \tb{locally finite} iff each point $u\in S$ has a neighborhood U such that U intersects only a finite
number of $U_\alpha$. A space is called \tb{paracompact} iff every open covering of S has a locally finite refinement of open sets, and S is
Hausdorff.
\end{definition}
\begin{thm}
Second countable, locally compact Hausdorff spaces are paracompact.
\end{thm}
\begin{proof}
S is the countable union of open sets $U_n$ s.t. $\overline{U_n}$ is compact and $\overline{U_n}\subset U_{n+1}$. If $W_\alpha$ is a covering
of S by open sets, and $K_n=\overline{U_n}-U_{n-1}$ then we can cover $K_n$ by a finit number of open sets each of which is contained in some
$W_\alpha\cap U_{n+1}$, and is disjoint from $\overline{U_{n-2}}$. The union of such collections yiels the desired refinement of $\{W_\alpha\}$.
\end{proof}
\begin{thm}
Every paracompact space is normal.
\end{thm}
\begin{proof}
We first show that if A is closed and $u\in A^c$, there are disjoint neighborhoods of u and A. For each $v\in A$ let $U_u, V_v$ be disjoint
neighborhood of u and v. Let $W_\alpha$ be a locally fninit refinement of the covering $V_v, A^c$ and $V=\cup W_\alpha$, the union over those
$\alpha$ so $W_\alpha\cap A\neq \emptyset$. A neighborhood $U_0$ of u meets a finite number of $W_\alpha$. Let U denote the intersection of
$U_0$ and the corresponding $U_u$. Then V and U are the required neighborhoods.
\end{proof}
\begin{thm}
If S is a Hausdorff space, the following are equivalent:
\begin{enumerate}
\item[(i)] S is normal;
\item[(ii)] For any two closed nonempty disjoint set A, B ther is a continuous function $f:S\rightarrow [0,1]$ s.t. $f(A)=0,f(B)=1$.(Urysohn's Lemma)
\item[(iii)] For any closed set $A\subset S$ and continuous function $f:A\rightarrow [a,b]$, there is a continuous extension $\tilde{f}:S
\rightarrow [a,b]$ of f (Tietze extension theorem)
\end{enumerate}
\end{thm}
\begin{definition}
The \tb{support} of $f:s\rightarrow \mathbb{R}$ is $supp(f)=\overline{\lbrace x\in S\mid f(x)\neq 0\rbrace}$. A \tb{partition of unity} on S
is a familiy of continuous mappings $\lbrace \varphi_i:S\rightarrow [0,1]\rbrace$ s.t.
\begin{enumerate}
\item[(i)] $\lbrace supp(\varphi_i)\rbrace$ is locally finite.
\item[(ii)] $\sum_i\varphi_i(x)=1$ for all x.
\end{enumerate}
We say that a pratition of unity $\{\varphi_i\}$ is \tb{subordinate} to a covering $\{A_\alpha\}$ of S if $supp(\varphi_i)$ is a refrinement
of $\{A_\alpha\}$.
\end{definition}
\begin{thm}
Let S be paracompact and $\{U_i\}$ be any open covering of S. Then ther is a partition of unity $\{\varphi_i\}$ subordinate to $\{U_i\}$.
\end{thm}
\begin{definition}
A topological space S is \tb{connected} if $\emptyset$ and S are the only subsets of S that are both open and closed. A subset of S is connected
iff it is connected in the relative topology. A \tb{component} A of S is a nonempty connected subsect of S s.t. the only connected subset of S 
containing A is A; S is called \tb{locally connected} iff each point x has an open neighborhood containing a connected neighborhood of x.
\end{definition}
\begin{prop}
A space S is not connected iff either of the following holds.
\begin{enumerate}
\item[(i)] There is a nonempty proper subset of S that is both open and closed.
\item[(ii)] S is the disjoint union of two nonempty open sets.
\item[(iii)] S is the disjoint union of two nonempty closed sets.
\end{enumerate}
\end{prop}
\begin{prop}
Let S be a connected space and $f:S\rightarrow \mathbb{R}$ be continuous. Then f assumes every value between any two values $f(u),f(v)$.
\end{prop}
\begin{proof}
Suppose $f(u)<a<f(v)$ and f doses not assume the value a. Then $U=\lbrace u_0\mid f(u_0)<a\rbrace$ is both open and closed.
\end{proof}
\begin{prop}
Let S be a topological space and $B\subset S$ be connected.
\begin{enumerate}
\item[(i)] if $B\subset A \subset \overline{B}$, then A is connected;
\item[(ii)] if $B_\alpha$ are connected and $B_\alpha\cap B\neq \emptyset$, then $B\cup\left(U_\alpha B_\alpha\right)$ is connected.
\end{enumerate}
\end{prop}
\begin{proof}
If A is not connected, A is the disjoint union of $U_1\cap A$ and $U_2\cap A$ where $U_1,U_2$ are open in S. Then $U_1\cap B\neq \emptyset,
U_2\cap B\neq\emptyset$, so B is not connected.
\end{proof}
\begin{definition}
An \tb{arc} $\varphi$ in S is a continuous mapping $\varphi:I=[0,1]\rightarrow S$. If $\varphi(0)=u,\varphi(1)=v$, we say $\varphi$ 
joins u and v; S is called \tb{arcwise connected} iff every two points in S can be joined by an arc in S. A space is called \tb{locally
arcwise connected} iff each point has an arcwise connected neighborhood.
\end{definition}
\begin{prop}
Every arcwise connected space is connected. If a space is connected and locally arcwise connected, it is arcwise connected.
\end{prop}
\begin{proof}
If S is arcwise connected and not connected, write $S=U_1\cup U_2$ where $U_1,U_2$ are nonempty, disjoint and open. Let $u_1\in U_1,
u_2\in U_2$ and let $\varphi$ be an arc joining $u_1,u_2$. Now $\varphi(I)$ is connected, and since $\varphi(I)\cap U_i\neq \emptyset$,
$\varphi\cap U_1 \cap U_2\neq\emptyset$. Hence $U_1\cap U_2\neq\emptyset$, a contradiction. Let $u\in S$and U denote all points that
can be joined to u by an arc. An easy argument shows U and $U^c$ are open and so $U=S$.
\end{proof}
\begin{definition}
Let S be a metric space with metric d, and $2^S$ denote the set of all subsets of S. Define $\tilde{d}(A,B)=sup{d(a,B)|a\in A}$. As
this is not symmetric, we further define $\overline{d}(A,B)=sup\lbrace\tilde{d}(A,B),\tilde{d}(B,A)\rbrace$. If $A\neq\emptyset,
B=\emptyset,\overline{d}(A,B)=\infty, \overline{d}(\emptyset,\emptyset)=0$. We call it the \tb{Hausdorff metric}.
\end{definition}
\begin{prop}
Let S be a metric space and d the Hausdorff metric on $2^S$. Then $f:S\rightarrow 2^S$ is continuous at $u_0\in S$ iff for all
$\varepsilon>0$ there is a $\delta>0$ s.t. $d(u,u_0)<\delta$ implies:
\begin{enumerate}
\item[(i)] for all $a\in f(u)$, there is a $b\in f(u_0)$ s.t. $d(a,b)<\varepsilon$; 
that is $$f(u)\subset \underset{b\in f(u_0)}{\cup}D_\varepsilon(b)$$.
\item[(ii)] for all $b\in f(u_0)$, there is an $a\in f(u)$ s.t. $d(b,a)<\varepsilon$.
\end{enumerate}
\end{prop}
\begin{definition}
Let S be a set. An \tb{equivalence relation} $\sim$ on S is a binary relation s.t. for all $u,v,w\in S$
\begin{enumerate}
\item[(i)] $u\sim u$;
\item[(ii)] $u\sim v$ iff $v\sim u$;
\item[(iii)] $u\sim v, v\sim w \Rightarrow u\sim w$.
\end{enumerate}
The \tb{equivalence class} containing u, denoted $[u]$ is defined by $[u]=\lbrace v\in S\mid u\sim v\rbrace$. The set of equivalence
classes is denote $S/\sim$, and the mapping $\pi:S\rightarrow S/\sim$; $u\longmapsto [u]$ is called the \tb{canonical projection}.
\end{definition}
\begin{definition}
$\lbrace U\subset S/\sim\mid \pi^{-1}(U) \text{is open in} S\rbrace$ is called the \tb{quotient topology}.
\end{definition}
\begin{example}
Consider $\R{2}$ and the relation $\sim$ defined by $(a_1,a_2)\sim (b_1,b_2)$ iff $a_1-b_1,a_2-b_2\in \Z$. Then $T^2=\R{2}/\sim$
is called the 2-\tb{torus}. In addition to the quotient topology, it inherits a group structure in the usual way: $[(a_1,a_2)]+
[(b_1,b_2)]=[(a_1,a_2)+(b_1,b_2)]$.
\end{example}
\begin{example}
The Klein bottle is obtained by reversing one of the orientations. Notice that $K^2$ is not "orientable" and does not inherit a 
group structure from $\R{2}$.
\end{example}
\begin{definition}
Let Z be a topological space and $c:[0,1]\rightarrow Z$ a continuous map s.t. $c(0)=c(1)=p\in Z$. We call c a \tb{loop} in Z
based at p. The loop c is called \tb{contractible} if there is a continuous map $H:[0,1]\times [0,1]\rightarrow Z$ s.t. $H(t,0)=
c(t),H(t,1)=p, \forall t\in [0,1]$.
\end{definition}
\begin{definition}
A space Z is called \tb{simply connected} iff every loop in Z is contractible.
\end{definition}
\begin{definition}
Let X be a topological space and $A\subset X$. Then A is called $residual$ iff A is the intersection of a countable family of
open dense subsets of X. A space X is called a \tb{Baire space} iff every residual set is dense.
\end{definition}
\begin{lem}
Let X be a locally Baire space; that is, each point $x\in X$ has a neighborhood U s.t. $\overline{U}$ is a Baire space. Then X 
is a Baire space.
\end{lem}
\begin{proof}
Let $A\subset X$ be residual; $A=\cap_1^\infty O_n$ where $\overline{O_n}=X$. Then $A\cap \overline{U}=\cap_1^\infty (O_n\cap
\overline{U})$ Now $O_n\cap \overline{U}$ is dense in $\overline{U}$ for if $u\in \overline{U}, u\in O$ then $O\cap U\neq\emptyset,
O\cap U \cap O_n\neq\emptyset$. Hence $\overline{U}\subset \overline{A},  \overline{A}=X$.
\end{proof}
\begin{thm}[\tb{Baire category}]
Every complete pseudometric space is a Baire space.
\end{thm}
\begin{proof}
Let $U\subset X$ be open and $A=\cap_1^{\infty}O_n$ be residual. We must show $U\cap A\neq\emptyset$. Now as $\overline{O_n}=X,
U\cap O_n\neq\emptyset$ and so we can choose a disk of diameter less than one, say $V_1$, s.t. $\overline{V_1}\subset U\cap O_1$.
Proceed inductively to obtain $\overline{V_n}\subset U\cap O_n\cap V_{n-1}$, where $V_n$ has diameter $<1/n$. Let $x_n\in 
\overline{V_n}$. Clearly $\{x_n\}$ is a Cauchy sequence, and by completeness has a convergence subsequence with limit point x.
Then $x\in \cap_1^{\infty}\overline{V_n}, U\cap \cap_1^{\infty}O_n\neq\emptyset$.
\end{proof}
\section*{Exercises}
\begin{exer}
Let S and T be sets and $f:S\rightarrow T$. Show that f is a bijection iff there is a mapping $g:T\rightarrow S$ s.t. $f\circ g,
g\circ f$ are identity mappings.
\end{exer}
\begin{exer}
Let X and Y be topological space with Y Hausdorff. Then show that, for any continuous maps $\map{f,g}{X}{Y},\lbrace x\in X\mid
f(x)=g(x)$ is closed. [Hint: Consider the mapping $x\longmapsto (f(x),g(x))$].
\end{exer}
\begin{exer}
Prove that in a Hausdorff space, single points are closed.
\end{exer}
\begin{exer}
Define a topological manifold as a space locally homeomorphic to $\R{n}$. Find a topological manifold that is not Hausdorff and 
not locally compact. [Hint: Consier $\mathbb{R}\cup\{\pm\infty\}$].
\end{exer}
\begin{exer}
Show that the continuous image of a connected space is connected.
\end{exer}
\section{Finite-Dimensional Banach Sapce}
\begin{definition}
A \tb{norm} on a vector space E is a mapping $\parallel\bullet\parallel:E\rightarrow \mathbb{R}$ s.t.
\begin{enumerate}
\item[(N1)] $\parallel\bullet\parallel\geq 0, \forall e\in E$ and $\parallel e\parallel=0$ iff $e=0$.
\item[(N2)] $\parallel\lambda e\parallel=|\lambda|\parallel e\parallel, \forall e\in E, \lambda\in\mathbb{R}$.
\item[(N3)] $\parallel e_1+e_2\parallel\leq\parallel e_1\parallel+\parallel e_2\parallel, \forall e_1,e_2\in E$. 
\end{enumerate}
\end{definition}
A normed space whose induced metric is complete is a \tb{Banach space}.
\begin{definition}
Two norms on a vector space E are \tb{equivalent} iff they induce the same topology on E.
\end{definition}
\begin{thm}
Let E be a finite-dimensional real vector space. Then
\begin{enumerate}
\item[(i)] there is a norm on E;
\item[(ii)] all norms on E are equivalent;
\item[(iii)] all norms on E are complete.
\end{enumerate}
\end{thm}
\begin{thm}
For finite-dimensional real vector spaces, linear and multilinear maps are continuous.
\end{thm}
\begin{cor}
Addition and scalar multiplication in a vector space are continuous maps from $E\times E\rightarrow E, \mathbb{R}\times E
\rightarrow E$.
\end{cor}
\begin{definition}
Given E,F we let $L(E,F)$ denote the set of all linear maps from E into F together with the natural structure of finite
dimensional real vector space. Similarly, $L^k(E,F)$ denote the space of multilinear maps from $E\times\cdots\times E$ into
F, $L_s^k(E,F)$, the subspace of symmetric elements of $L^k(E,F)$ [that is, if $\pi$ is any permutation of $\{1,2\cdots,k\}$,
we have $f(e_1,\cdots,e_2)=f(e_{\pi(1)},\cdots,e_{\pi(k)})$] and $L_a^k(E,F)$ the subspace of skew symmetric [That is 
if $\pi$ is any permutation of $\{1,2\cdots,k\}$,we have $f(e_1,\cdots,e_2)=(sgn \pi)f(e_{\pi(1)},\cdots,e_{\pi(k)})$, 
wher $sgn \pi = \pm 1$ according as $\pi$ is an even or odd permutation].
\end{definition}
\begin{thm}
There is a natural isomorphism $L(E,L^k(E,F))\approx L^{k+1}(E,F)$.
\end{thm}
\section*{Exercises}
\begin{exer}
Let $f\in L(E,F)$ so that f is continuous.
\begin{enumerate}
\item[(a)] Show that there is a constant K s.t. $\parallel f(e)\parallel\leq K\parallel e\parallel$ for all $e\in E$.
Define $\parallel f\parallel$ as the greatest lower bound of such K.
\item[(b)] Show that this is a norm on $L(E,F)$.
\item[(c)] Prove that $\parallel f\circ g\parallel\leq \parallel f\parallel \cdot \parallel g\parallel$.
\end{enumerate}
\end{exer}
\begin{exer}
Suppose $f\in L(E,F)$ and $dim(E)=dim(F)$. Then f is an isomorphism iff it is a monomorphism (one-to-one) and iff it is
surjective (onto).
\end{exer}
\begin{exer}
Show that two norms $\parallel\bullet\parallel_1,\parallel\bullet\parallel_2$ are equivalent iff there is a constant M
s.t. $M^{-1}\parallel e\parallel_1\leq \parallel e\parallel_2\leq M \parallel e\parallel_1$.
\end{exer}
\begin{exer}
Let E be the set of all $C^1$ functions $\map{f}{[0,1]}{\mathbb{R}}$ with the norm $\parallel f\parallel=sup_{x\in
[0,1]}|f(x)|+ sup_{x\in [0,1]}|f'(x)|$. Prove that E is a Banach space.
\end{exer}
\section{Local Differential Calculus}
\begin{definition}
Let E,F be two vector spaces with maps $\map{f,g}{U\subset E}{F}$. We say f and g are \tb{tangent} at $u_0\in U$ iff
$$ \lim_{u\rightarrow u_0}\frac{\norm{f(u)-g(u)}}{\norm{u-u_0}}=0$$
\end{definition}
\begin{thm}
For $\map{f}{U\subset E}{F}$ there is at most one $L\in L(E,F)$ so that the map $\map{g_L}{U\subset E}{F}$ given by
$g_L(u)=f(u_0)+L(u-u_0)$ is tangent to f at $u_0$.
\end{thm}
\begin{definition}
If there is such an L we say f is \tb{differentiable} at $u_0$, and define the \tb{derivative} of f at $u_0$ to be
$Df(u_0)=L$. If f is differentiable at each $u\in U$, the map $\map{\mathrm{D}f}{U}{L(E,F)};u\mapsto \mathrm{D}f(u)$ 
is the \tb{derivative} of f. Moreover, if $\mathrm{D}f$ is a continuous map we say f is of class $C^1$.
\end{definition}
\begin{definition}
Suppose $\map{f}{U\subset E}{F}$ is of class $C^1$. Define the \tb{tangent} of f to be the map: $Tf:U\times E\rightarrow
F\times F$ given by $Tf(u,e)=(f(u),Df(u)\cdot e$.
\end{definition}
From a geometrical point of view, T is more natural than D. If we take $(u,e)$ as a vector with base point u, then $(f(u),
Df(u)\cdot e)$ is the image vector with its base point. Another reason for this is its behavior under composition, so T is
a covariant functor.
\begin{thm}
\begin{align*}
T(g\circ f)&=T(g)\circ T(f) \\
T^r(g\circ f)&=T^r(g)\circ T^r(f)
\end{align*}
\end{thm}
For $\map{f}{E}{F},\map{c}{I}{U}$. So $Df(u)\cdot e=\frac{\text{d}}{\text{d}t}\lbrace f(u+te)\rbrace\mid_{t=0}$. $Df(u)$ is
represented by the usual Jacobian matrix. If we apply the fundamental theorm of calculus to $t\mapsto f(tx+(1-t)y)$ and 
$\norm{Df(tx+(1-t)y)}\leq M$, we obtain the mean value inequality:$\norm{f(x)-f(y)}\leq M\norm{x-y}$.
\begin{definition}
Let $U_1\subset E_1, U_2\subset E_2$ be open and suppose $\map{f}{U_1\times U_2}{F}$. Then the \tb{partial derivative} of f
with respect to $E_1$ denoted $D_1f$ is defined by $D_1f(u_1,u_2):E_1\rightarrow F: e_1\mapsto D_1f(u_1,u_2)\cdot e_1 = 
Df(u_1,u_2)\cdot (e_1,0)$. Thus $Df=D_1f+D_2f$.
\end{definition}
\begin{thm}[\tb{Inverse Mapping}]
Let $\map{f}{U\subset E}{F}$ be of class $C^r$ and suppose $Df(x_0)$ is a linear isomorphism. Then f is a $C^r$ diffeomorphism 
of some neighborhood of $x_0$ onto some neighborhood of $f(x_0)$. 
\end{thm}
\begin{lem}
Let M be a complete metric space, Let $\map{F}{M}{M}$ and assume there is a constant $0\leq\lambda<1$ s.t. $\forall x,y\in M,
d(F(x),F(y))\leq \lambda d(x,y)$. Then F has a unique fixed point $x_0\in M, F(x_0)=x_0$.
\end{lem}
\begin{proof}
Pick $x_1\in M$ and define $x_{n+1}=F(x_n)$. Thus $d(x_{n+1},x_n)\leq \lambda^{n-1}d(F(x_1),x_1)$ and $d(x_{n+k},x_n)\leq 
\left(\sum_{j=n-1}^{n+k-1}\lambda^{j}\right)d(F(x_1),x_1)$. Thus $x_n$ is a Cauchy sequence. Since F is obviously uniformly
continuous, then $x_0=\lim_{n\rightarrow \infty}x_n=\lim_{n\rightarrow \infty}x_{n+1}=\lim_{n\rightarrow \infty}F(x_n)=F(x_0)$.
\end{proof}
\begin{thm}[\tb{Implicit Function}]
Let $U\subset E, V\subset F$ and $\map{f}{U\times V}{G}$ be $C^r$. For some $x_0,y_0$ assume $D_2f(x_0,y_0):F\rightarrow G$ 
is an isomorphism. Then there are neighborhoods $x_0\in U_0, f(x_0,y_0)\in W_0$ and a unique $C^r$ map $\map{g}{U_0\times W_0}{V}$
s.t. $\forall (x,w)\in U_0\times W_0, f(x,g(x,w))=w$.
\end{thm}
\begin{proof}
Consider the map $\Phi:(x,y)\mapsto (x,f(x,y))$, then $$D\Phi(x_0,y_0)\cdot(x_1,y_1)=
\left(\begin{matrix}
I & 0 \\
D_1f(x_0,y_0) & D_2f(x_0,y_0)
\end{matrix}\right)
\left(\begin{matrix}
x_1 \\
y_1
\end{matrix}\right)$$
which is easily seen to be an isomorphism. Thus $\Phi$ has a unique $C^r$ local inverse $\Phi^{-1}:(x,w)\mapsto (x,g(x,w))$.
Then g so defined is the desired map.
\end{proof}
\section*{Exercises}
\begin{exer}
Show that $$T^2f:(U\times E)\times(E\times E)\rightarrow (F\times F)\times F\times F $$ $$ (u,e_1,e_2,e_3)\mapsto 
(f(u),Df(u)\cdot e_1, Df(u)\cdot e_2, D^2f(u)\cdot(e_1,e_2)+Df(u)\cdot e_3$$
\end{exer}
\begin{exer}
Develop a formula for $D^r(f\circ g), D^r(fg)$.
\end{exer}
\section{Manifolds and Mappings}
\begin{definition}
A \tb{local chart} on S is a bijection $\varphi$ from a subsut U of S to an open subset of some (fimite-dimemtional, real)
vector space F, denoted as $(U,\varphi)$. An \tb{atlas} on S is a family $\mca{A}$ of charts $\{(U_i,\varphi_i)\}$ s.t.
\begin{enumerate}
\item[(1)] $S=\cup U_i$;
\item[(2)] Any two charts are compatable in the sense that the overlap maps between members of $\mca{A}$ are $C^\infty$ 
diffeomorphisms. 
\end{enumerate}
\end{definition}
Two atlases $\mca{A}_1,\mca{A}_2$ are \tb{equivalent} iff $\mca{A}_1\cup \mca{A}_2$ is an atlas. A \tb{differentiable structure}
$\mca{S}$ on S is an equivalence class of atlases on S. The union of the atlases $\mca{A_g}=\cup \mca{A}$ is the \tb{maximal atlas}.
A \tb{differentiable manifold} M is a pair $(S,\mca{S})$. A manifold will always mean a Hausdorff, second countable, differentiable
manifold.
\begin{definition}
Let $(S_1,\mca{S}_1), (S_2,\mca{S}_2)$ be two manifolds. The \tb{product manifold} is $(S_1\times S_2, \mca{S}_1\times \mca{S}_2)$.
\end{definition}
\begin{definition}
A \tb{submanifold} of a manifold M is a subset $B\subset M$ with the property that for each $b\in B$ there is admissible chart 
$(U,\varphi)$ in M with $b\in U$ which has the \tb{submanifold property}, $\varphi:U\rightarrow E\times F$, and $\varphi(U\cap B)
=\varphi(U)\cap (E\times\{0\})$. And its differentiable structure generated by the atlas is $\lbrace(U\cap B, \varphi\mid U\cap 
B)\rbrace$.
\end{definition}
\begin{definition}
Suppose we have $\map{f}{M}{N}$ and charts $(U,\varphi), (V,\phi)$. So the \tb{local representative} of f, $f_{\varphi\phi}=
\phi\circ f\circ\varphi^{-1}$.
\end{definition}
\begin{definition}
A map $\map{f}{M}{N}$ is called a diffeomorphism if f is of class $C^r$, is a bijection, and $f^{-1}$ is of class $C^r$.
\end{definition}
\section*{Exercises}
\begin{exer}
Prove that $S^1$ is a submanifold of $\R{2}$.
\end{exer}
\section{Vector Bundles}
\section*{Exercises}
\section{The Tangent Bundle}
\section*{Exercises}
\section{Tensors}
\section*{Exercises}